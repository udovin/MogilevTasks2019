\begin{problem}{Вр\'{е}менные трудности}{}{}{}{}

Крупная сталелитейная компания <<Rearden~Steel>> испытывает трудности в получении сырья для производства. Многие рудники закрылись, оставшиеся оказались ненадежными и не выполняли заказы вовремя. Правительство пообещало, что все трудности временные и скоро ситуация улучшится, но заводы <<Rearden~Steel>> не могут простаивать. В связи с этим компания решила открыть свои рудники.

Для этого <<Rearden~Steel>> выкупила крупный участок земли размером $n\times m$ квадратных метров. Для удобства разделим участок на квадраты размером один на один метр. Оказалось, что на каждый такой квадрат приходится либо одно месторождение меди, либо одно месторождение железной руды. На этом участке и планируется размещать рудники, которые будут иметь прямоугольную форму, а их границы будут параллельны сторонам участка. Границы рудника таже должны проходить по линиям, разделяющим квадраты. Также, очевидно, никакие два рудника не должны пересекаться.

Согласно последней Директиве правительства, компаниям запрещено строить рудники размером более $h$ квадратных метров. Однако разрешено (пока что) строить несколько рудников. Кроме того, на территории рудника должно содержаться как минимум $l$ квадратов каждой из руд, иначе он будет считаться неприбыльным.

Теперь руководитель компании, мистер Риарден, хочет построить на этом участке несколько рудников таким образом, чтобы было занято как можно больше клеток. Он нанял Вас, чтобы Вы нашли оптимальный способ расположения рудников согласно правилам, описанным выше, а также согласно Директиве правительства.

\InputFile

Входные данные находятся в файлах \texttt{input1.txt}, \texttt{input2.txt}, ... , \texttt{input10.txt}. Каждый файл соответствует описанию одного участка земли.

В первой строке входных данных находится целое число $t$~--- номер теста.

Во второй строке входных данных находятся четыре целых числа $n$, $m$, $l$ и $h$~--- размеры участка земли, минимальное количество каждой из руд в одном руднике и максимальный размер рудника.

В каждой из следующих $n$ строк находится строка, состоящая из $m$ символов. Если $j$-й символ $i$-й строки равен <<\t{F}>>, то в клетке $(i, j)$ содержится железная руда. Если же этот символ равен <<\t{C}>>, то в клетке $(i, j)$ содержится медь.

\OutputFile

В первой строке выходных данных должно находиться целое число $k$ ($0 \le k \le n\cdot m$)~--- количество рудников, которые должен открыть <<Rearden Steel>>.

В каждой из следующих $k$ строк должно содержаться четыре целых числа $x_{1k}$, $y_{1k}$, $x_{2k}$ и $y_{2k}$ ($1 \le x_{1k} \le x_{2k} \le n$, $1 \le y_{1k} \le y_{2k} \le m$)~--- координаты верхнего левого и правого нижнего углов $i$-го рудника.

\Examples

\begin{example}
\exmp{
0
3 5 1 6
FFFFF
FCCCF
FFFFF
}{
3
1 1 3 2
1 3 3 3
1 4 3 5
}%
\end{example}

\Scoring

Если выходной файл не соответствует указанному формату выходных данных, то Вы получите $0$ баллов за тест.

Если найденное Вами расположение рудников не соответствует условию задачи, то Вы получите $0$ баллов за тест.

Если какой-либо рудник имеет площадь более $h$ квадратных метров, то это нарушение Директивы правительства. В этом случае Вы получите $0$ баллов за тест.

Иначе Ваш балл за тест равен \[10\cdot\left(\frac{Q}{nm}\right)^3,\] где $n$ и $m$~--- размеры поля, а $Q$~--- количество клеток, занятое Вашим решением.

В задаче всего десять тестов, Ваш балл за задачу равен сумме баллов по всем тестам. То есть, Вы можете набрать не более чем $10\cdot10 = 100$ баллов за эту задачу.

Гарантируется, что по крайней мере на пяти тестах существует оптимальное решение, дающее $10$ баллов за тест.

\end{problem}

