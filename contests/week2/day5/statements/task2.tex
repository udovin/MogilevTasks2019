\begin{problem}{Пакетный менеджер}{input.txt}{output.txt}{2 секунды}{256 мегабайт}

Мистер APT работает пакетным менеджером на складе debian-пакетов. В его распоряжении имеется $n$ пакетов, при этом $i$-й пакет имеет массу $a_i$ байт. Debian-пакеты легко копируются, поэтому мистер APT может копировать пакет сколько угодно раз, создавая новый пакет с такой же массой.

Задача пакетного менеджера~--- выдавать пакеты клиентам. Сегодня у мистера APT очень важный клиент~--- Одрапег Александрович. Клиент хочет заказать некоторый набор debian-пакетов с заданной массой $m$. Одрапег Александрович пока не определился со значением $m$, но сообщил, что оно будет целым, и при этом $1 \le m \le x$.

Мистер APT боится, что точное значение $m$ будет известно в самый последний момент, поэтому ему необходимо выбрать такой набор debian-пакетов, чтобы можно было в любом случае выполнить заказ Одрапега Александровича, используя только пакеты из этого набора. Распаковывать debian-пакеты при этом не разрешается, так как при этом они сильно увеличиваются в объеме. Количество debian-пакетов в наборе необходимо минимизировать.

Напоминаем, что debian-пакеты разрешается клонировать, но ровно до того момента, как будет выбран требуемый набор.

\InputFile

В первой строке входных данных находится целое число $n$ ($1 \le n \le 10^5$)~--- количество debian-пакетов на складе.

Во второй строке входных данных находится $n$ целых чисел $a_i$ ($1 \le a_i \le 10^{12}$)~--- масса $i$-го debian-пакета на складе.

В третьей строке входных данных находится целое число $x$ ($1 \le x \le 10^{12}$)~--- максимальное значение суммарной массы $m$, которое может потребовать Одрапег Александрович.

\OutputFile

Выведите одно целое число $z$~--- минимальное количество пакетов в требуемом наборе. Если такой набор выбрать невозможно, выведите целое отрицательное число $-1$. Если Вы не умеете решать задачу, выведите целое отрицательное число $-42$, но баллов за это Вы не получите, конечно же.

\Examples

\begin{example}
\exmp{
2
2 1
3
}{
2
}%
\exmp{
1
1
1
}{
1
}%
\exmp{
4
5 2 1 3
15
}{
5
}%
\exmp{
2
5 3
2
}{
-1
}%
\end{example}

\Notes

В первом примере мистер APT может выбрать debian-пакеты $a_1$ и  $a_2$:
\begin{itemize}
\item При $m = 1$ он отдаст один debian-пакет $a_2$.
\item При $m = 2$ он отдаст один debian-пакет $a_1$.
\item При $m = 3$ он отдаст два debian-пакета $a_1, a_2$ ($2 + 1 = 3$).
\end{itemize}

Во втором примере мистер APT выберет один debian-пакет $a_1$.

\bigskip

В третьем примере мистер APT может выбрать debian-пакеты $\{a_1, a_1, a_2, a_2, a_3\}$:
\begin{itemize}
\item При $m = 1$ он отдаст debian-пакет $a_3$.
\item При $m = 2$ он отдаст debian-пакет $a_2$.
\item При $m = 3$ он отдаст debian-пакеты $a_2, a_3$.
\item При $m = 4$ он отдаст debian-пакеты $a_2, a_2$.
\item При $m = 5$ он отдаст debian-пакеты $a_2, a_2, a_3$.
\item При $m = 6$ он отдаст debian-пакеты $a_1, a_3$.
\item При $m = 7$ он отдаст debian-пакеты $a_1, a_2$.
\item При $m = 8$ он отдаст debian-пакеты $a_1, a_2, a_3$.
\item При $m = 9$ он отдаст debian-пакеты $a_1, a_2, a_2$.
\item При $m = 10$ он отдаст debian-пакеты $a_1, a_1$.
\item При $m = 11$ он отдаст debian-пакеты $a_1, a_1, a_3$.
\item При $m = 12$ он отдаст debian-пакеты $a_1, a_1, a_2$.
\item При $m = 13$ он отдаст debian-пакеты $a_1, a_1, a_2, a_3$.
\item При $m = 14$ он отдаст debian-пакеты $a_1, a_1, a_2, a_2$.
\item При $m = 15$ он отдаст debian-пакеты $a_1, a_1, a_2, a_2, a_3$.
\end{itemize}
Обратите внимание, что в данном тесте есть и другие варианты выбрать debian-пакеты, например, $\{a_1, a_1, a_3, a_3, a_4\}$.

\bigskip

В четвертом примере мистер APT не может выбрать debian-пакеты таким образом, чтобы суммарная масса была равна $2$.

\Scoring

В этой задаче \textbf{потестовая оценка}. Тесты условно разбиты на подзадачи, за полное прохождение всех тестов подзадачи начисляются соответствующие ей баллы. Подзадачи приведены в следующей таблице:

\medskip

\begin{tabular}{| c | c | c |} \hline
	№ & Ограничения & Баллы за подзадачу \\ \hline
	1 & $n = 1$, $a_i \le 25, x \le 25$ & 8 \\ \hline
	2 & $n \le 3$, $a_i \le 25, x \le 25$ & 12 \\ \hline
	3 & $n \le 5$, $a_i \le 25, x \le 25$ & 12 \\ \hline
	4 & $a_i \le 10^5$, $x \le 10^5$ & 28 \\ \hline
	5 & Нет дополнительных ограничений & 40 \\ \hline
\end{tabular}

\end{problem}

