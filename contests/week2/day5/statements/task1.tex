\begin{problem}{Рисование окон}{input.txt}{output.txt}{1 секунда}{256 мегабайт}

Дракончик Konqi рисует картину, на которой будут изображены окна.

Картина будет огромной, поэтому в одиночку ему не справиться. Konqi попросил помощи у гномов. $n$ гномов согласилось помочь ему. Известно, что за час $i$-й гном может нарисовать ровно $a_i$ окон.

Работать над картиной большим коллективом неудобно, поэтому Konqi хочет выбрать ровно двух гномов себе в помощники. Понятно, что если он выберет $i$-го и $j$-го гномов, в час они вместе нарисуют $a_i + a_j$ окон.

Конечно же, оптимальным решением было бы взять самых быстрых гномов, но Konqi хочет при этом получить побольше удоволствия от процесса рисования. Поэтому он хочет выбрать $k$-ю пару гномов, если эти пары отсортировать в порядке неубывания количества окон, которое они нарисуют за час.

Более формально, сформируем массив $b$ из всех $\frac{n\cdot(n-1)}2$ сумм $a_i + a_j$ по всем возможным парам гномов. Затем отсортируем этот массив по неубыванию и найдем значение $b_k$. Это число и интересует дракончика.

Вам необходимо помочь Konqi и помочь ему выбрать нужную пару гномов.

\vspace{-3.5em}

\begin{center}
	\includegraphics[width=0.35\textwidth]{task1.png}
	
	\scriptsize \textit{Источник:} \t{https://community.kde.org/File:Mascot\_konqi-app-graphics.png}
\end{center}

\InputFile

В первой строке входных данных находится два целых числа $n$ ($1 \le n \le 5\cdot10^5$, $1 \le k \le \frac{n(n-1)}2$)~--- количество гномов, согласившихся помочь Konqi, и номер пары гномов, которую хочет выбрать дракончик.

Во второй строке входных данных находится $n$ целых чисел $a_i$ ($1 \le a_i \le 10^9$)~--- количество окон, которое может нарисовать $i$-й гном за час.

\OutputFile

Выведите одно целое число~--- ответ на задачу. Если он окажется неправильным, Konqi огорчится. Не расстраивайте дракончика!

\Examples

\begin{example}
\exmp{
3 3
7 1 4
}{
11
}%
\exmp{
5 7
1 5 3 5 3
}{
8
}%
\exmp{
9 15
6 7 4 1 1 5 2 5 2
}{
7
}%
\end{example}

\Scoring

В этой задаче \textbf{потестовая оценка}. Тесты условно разбиты на подзадачи, за полное прохождение всех тестов подзадачи начисляются соответствующие ей баллы. Подзадачи приведены в следующей таблице:

\medskip

\begin{tabular}{| c | c | c |} \hline
	№ & Ограничения & Баллы за подзадачу \\ \hline
	1 & $n \le 10^4$ & 5 \\ \hline
	2 & $a_i \le 5000$ & 15 \\ \hline
	3 & $k \le 10^6$ & 30 \\ \hline
	4 & $n \le 10^5$ & 25 \\ \hline
	5 & Нет дополнительных ограничений & 25 \\ \hline
\end{tabular}

\end{problem}

