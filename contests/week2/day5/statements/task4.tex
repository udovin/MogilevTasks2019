\begin{problem}{Криптографическая защита}{input.txt}{output.txt}{4 секунды}{256 мегабайт}

Во многих криптографических протоколах используется техногология \textit{Challenge-Based Authentification}, позволяющая входить в систему без передачи пароля по открытому каналу. Принцип ее работы таков, что сервер задает клиенту некоторый вопрос, на который можно ответить, только имея правильный пароль либо ключ. Обычно в качестве вопроса сервер запрашивает некоторую функцию от криптографического ключа клиента.

Создатели одного известного сайта с задачами по информатике, \texttt{upsolve.by}, решили внедрить эту технологию аутентификации. Они подошли к делу довольно оригинально и сделали настоящий challenge: для успешного входа в систему необходимо правильно решить задачу.

Мальчик Антоша любит тренироваться на \t{upsolve.by}. Но задача, которую необходимо решить для входа на сайт, оказалась слишком сложной для него. Задача выглядит так:

\begin{quote}
\it
Определим последовательность Фибоначчи следующим образом:

\begin{itemize}
	\item $F_1 = 1$
	\item $F_2 = 2$
	\item $F_n = F_{n-1} + F_{n-2}$ для $n \ge 3$
\end{itemize}

Первые несколько элементов этой последовательности: $1, 2, 3, 5, 8, 13, 21, \dots$.

Для целого положительного числа $p$, пусть $X(p)$ будет равно числу различных способов выразить $p$ в виде суммы различных чисел Фибоначчи. Два способа считаются различными, если существует число Фибоначчи, которое содержится в одном из них, но не содержится в другом.

Вам дана последовательность из $n$ целых положительных чисел $a_1, a_2, \dots, a_n$. Для каждого непустого префикса $a_1, a_2, \dots, a_k$ введем обозначение $p_k = F_{a_1} + F_{a_2} + \dots + F_{a_k}$. Ваша задача~--- найти $X(p_k)$ по модулю $10^9 + 7$ для всех $k = 1, 2, \dots, n$.
\end{quote}

Помогите Антоше решить эту непростую задачу!

\begin{center}
	\includegraphics[width=0.5\textwidth]{task4.jpg}
	
	\scriptsize \textit{Источник:} \t{https://xkcd.ru/538/}
\end{center}

\InputFile

В первой строке входных данных находится целое число $n$ ($1 \le n \le 10^5$)~--- длина последовательности.

Во второй строке входных данных находится $n$ целых чисел $a_i$ ($1 \le a_i \le 10^9$)~--- числа в последовательности.

\OutputFile

Выведите $n$ целых чисел, по одному числу в каждой строке. $i$-я строка должна содержать значение $X(p_i)$ по модулю $10^9 + 7$.

\Examples

\begin{example}
\exmp{
4
4 1 1 5
}{
2
2
1
2
}%
\end{example}

\Notes

Рассмотрим пример.

Число $5$ можно выразить двумя способами:

\begin{itemize}
\item $F_2 + F_3$ ($2 + 3$)
\item $F_4$ ($5$)
\end{itemize}

Следовательно, $X(p_1) = 2$.

Затем имеем $X(p_2) = 2$, поскольку $p_2 = 1 + 5 = 1 + 2 + 3$.

Единственный способ представить $7$ в виде суммы различных чисел Фибоначчи~--- $2 + 5$.

Наконец, $15$ можно выразить как $2 + 13$ и $2 + 5 + 8$ (два способа).

\Scoring

В этой задаче \textbf{оценка по подзадачам}. Тесты условно разбиты на подзадачи, за полное прохождение всех тестов подзадачи начисляются соответствующие ей баллы. Если какой-то тест подзадачи не пройден, баллы за нее не начисляются. Подзадачи приведены в следующей таблице:

\medskip

\begin{tabular}{| c | c | c |} \hline
	№ & Ограничения & Баллы за подзадачу \\ \hline
	1 & $n, a_i \le 15$ & 5 \\ \hline
	2 & $n, a_i \le 100$ & 20 \\ \hline
	3 & $n \le 100$, все $a_i$~--- квадраты различных целых чисел & 15 \\ \hline
	4 & $n \le 100$ & 10 \\ \hline
	5 & Все $a_i$~--- различные четные числа & 15 \\ \hline
	6 & Нет дополнительных ограничений & 35 \\ \hline
\end{tabular}

\end{problem}

