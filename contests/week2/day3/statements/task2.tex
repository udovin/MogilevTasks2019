\begin{problem}{Запросики}{input.txt}{output.txt}{2 секунды}{128 мегабайт}

Федеральная служба по надзору в сфере опросов, запросов, вопросов и допросов Контестана (\textit{Контестзапроснадзор}) ведет Реестр распространителей задач на запросы. Недавно они решили модернизировать компьютерную систему, которая отвечает за хранение Реестра. Модернизацию будет проводить известная компания <<Queries~\&~Requests,~Ltd.>>

Вы работате на <<Queries~\&~Requests,~Ltd.>>, поэтому Вам было поручено написать ПО для хранения Реестра. Требования к программе следующие. В Реестре хранятся записи о распространителях задач на запросы. Каждая запись имеет свой порядковый номер и информацию о распространителе, которая является целым числом. Изначально в системе хранится запись только об одном распространителе (о самом Контестзапроснадзоре), при этом эта запись имеет номер $1$, а информация о распространителе отсутствует. В систему могут прийти следующие запросы:

\begin{itemize}
\item Запросить информацию о распространителе по его индексу.
\item Изменить информацию о распространителе на некоторое число $c$.
\item Откатить информацию о распространителе на состояние, предшествующее последнему изменению. Откаты можно совершать до тех пор, пока информация о распространителе присутствует.
\item В случае, если откат был применен ошибочно, возможно его отменить. Система контроля хранит историю откатов каждого распространителя. При применении очередного отката в историю делается соответствующая запись. При отмене отката запись стирается. В случае изменения (не отмены отката) вся история откатов данного распространителя стирается. Отмену можно применять до тех пор, пока в истории по распространителю существуют записи.
\item И, наконец, можно клонировать распространителей. То есть создать нового распространителя с той же последовательностью изменений и историей откатов.
\end{itemize}

Вам необходимо написать программу, которая обрабатывает все запросы в Реестр.

\InputFile

В первой строке входных данных находится целое число $n$~--- количество запросов ($1 \le n \le 5\cdot 10^5$).

Каждая из следующих $n$ строк имеет один из перечисленных форматов:

\begin{itemize}
\item \t{info $c_i$ $p_i$}. Изменить информацию о распространителе $c_i$ на число $p_i$ ($1 \le p_i \le 10^9$).
\item \t{rollback $c_i$}. Откатить последнее изменение у распространителя с номером $c_i$.
\item \t{undo $c_i$}. Отменить последний откат у распространителя с номером $c_i$.
\item \t{clone $c_i$}. Клонировать распространителя с номером $c_i$.
\item \t{check $c_i$}. Вывести текущую информацию о распространителе с номером $c_i$.
\end{itemize}

Все команды корректны, в частности, к записи о распространителе, не содержащей информации, не применяется \t{rollback}. А также \t{undo} возможен только при непустой истории откатов. В запросах может фигурировать только уже существующий распространитель. Номера распространителям присваиваются в порядке их возникновения. Распространитель, с которого начинается ведение Реестра, имеет номер один.

\OutputFile

После каждого запроса \t{check $c_i$} в отдельной строке выведите результат. Если о распространителе не указано информации, выведите \t{null}, иначе~--- текущую информацию о распространителе.

\Examples

\begin{example}
\exmp{
9
info 1 5
info 1 7
rollback 1
check 1
clone 1
undo 2
check 2
rollback 1
check 1
}{
5
7
null
}%
\end{example}

\Scoring

В этой задаче \textbf{потестовая оценка}. Тесты условно разбиты на подзадачи, за полное прохождение всех тестов подзадачи начисляются соответствующие ей баллы. Подзадачи приведены в следующей таблице:

\medskip

\begin{tabular}{| c | c | c |} \hline
	№ & Ограничения & Баллы за подзадачу \\ \hline
	1 & $n \le 1000$, отсутствуют запросы \t{undo} & $10$ \\ \hline
	2 & $n \le 1000$ & $20$ \\ \hline
	3 & Отсутствуют запросы \t{clone} и \t{undo} & $10$ \\ \hline
	4 & Отсутствуют запросы \t{clone} & $10$ \\ \hline
	5 & Отсутствуют запросы \t{undo} & $20$ \\ \hline
	6 & Нет дополнительных ограничений & $30$ \\ \hline
\end{tabular}

\end{problem}

