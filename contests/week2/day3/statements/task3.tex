\begin{problem}{Вопросики}{input.txt}{output.txt}{1 секунда}{128 мегабайт}

Два известных хакера Контестана, \t{ZGViaWFu} и \t{dWJ1bnR1}, взломали компьютерную сеть Института исследований в сфере теории игр Контестана (\textit{Интеоригр}) и получили админский доступ к паблику Интеоригра в социальной сети ВКонтесте. Они узнали, что в этом паблике всего $n$ подсписчиков, которые расположены в списке в порядке их регистрации в соцсети. Для каждого подписчика известно $a_i$~--- количество друзей у него.

\t{ZGViaWFu} и \t{dWJ1bnR1} решили сыграть в следующую игру:

\begin{itemize}
   \item Они будут ходить по очереди, первым начинает \t{ZGViaWFu}.
   \item На каждом ходу игрок может забанить любое ненулевое количество подписчиков с начала либо с конца списка. Но нельзя забанить всех. Забаненные подписчики удаляются из списка.
   \item Игра закончится, когда в паблике останется ровно один подписчик. \t{ZGViaWFu} хочет минимизировать его количество друзей, а \t{dWJ1bnR1} хочет максимизировать это число.   
\end{itemize}

Пока \t{ZGViaWFu} обедает, \t{dWJ1bnR1} хочет забанить $k$ подписчиков заранее. Он может забанить любых подписчиков, не обязательно с начала или конца списка. При этом относительный порядок остальных подписчиков не изменится. После обеда они будут играть по правилам выше, но в паблике останется всего $n-k$ подписчиков.

\t{dWJ1bnR1} пока еще не определился, какое число $k$ выбрать, поэтому он задал Вам неколько вопросов $k_i$, $1 \le i \le q$. Вам необходимо сказать для каждого из $q$ чисел $k_1, k_2, \dots, k_q$, какое количество друзей будет у последнего оставшегося подписчика, если он заранее сотрет $k_i$ чисел, а затем оба игрока будут играть оптимально.

\InputFile
В первой строке находится целое число $n$ ($1 \le n \le 10^6$)~--- количество подписчиков в паблике Интеоригра.

Во второй строке находятся $n$ целых чисел $a_i$ ($1 \le a_i \le 10^6$)~--- количество друзей у $i$-го подписчика.

В третьей строке находится целое число $q$ ($1 \le q \le 10^6$)~--- количество вопросов хакера \t{dWJ1bnR1}.

В четвертой строке находятся $q$ целых чисел $k_i$ ($0 \le k_i \le n- 1$)~--- $i$-й вопрос.

\OutputFile

Выведите $q$ чисел через пробел. $i$-е из них является ответом на $i$-й вопрос хакера \t{dWJ1bnR1}.

\Examples

\begin{example}
\exmp{
4
1 4 2 3
4
0 1 2 3
}{
1 3 3 4
}%
\exmp{
3
5 5 5
3
0 1 2
}{
5 5 5
}%
\exmp{
6
2 7 5 4 8 10
3
3 5 2
}{
7 10 7
}%
\end{example}

\Notes

В первом тесте при $k = 3$ хакеру \t{dWJ1bnR1} выгодно стереть первое, второе и четвертое числа.

\Scoring

В этой задаче \textbf{потестовая оценка}. Тесты условно разбиты на подзадачи, за полное прохождение всех тестов подзадачи начисляются соответствующие ей баллы. Подзадачи приведены в следующей таблице:

\medskip

\begin{tabular}{| c | c | c |} \hline
	№ & Ограничения & Баллы за подзадачу \\ \hline
	1 & Примеры из условия & $6$ \\ \hline
	2 & $n \le 3$, $q = 1$, $k_1 = 0$ & $10$ \\ \hline
	3 & $n \le 100$, $q = 1$, $k_1 = 0$ & $20$ \\ \hline
	4 & $n \le 10^5$, $q \le 2$, $k_1 \le 1$ & $24$ \\ \hline
	5 & $n \le 10^5$, $q \le 10^5$ & $20$ \\ \hline
	6 & Нет дополнительных ограничений & $20$ \\ \hline
\end{tabular}

\end{problem}

