\begin{problem}{Опросики}{input.txt}{output.txt}{1 секунда}{128 мегабайт}

Национальный комитет по социологическиим опросам Контестана (\textit{Контестсоцопрос}) внедряет новую систему управления опросами. Систему разрабатывает крупная компания <<Queries~\&~Requests,~Ltd.>>

Вы работаете на эту компанию. Вам было поручено написать часть кода, которая отвечает за присвоение опросам порядковых номеров. Правила, по которым запросу присваивается номер, следующие: при добавлении нового запроса ему присваивается минимальный целый положительный номер, не занятый никаким из существующих опросов. Опросы также можно удалять. После удаления опроса присвоенный ему номер становится свободным и может быть занят каким-либо из последующих опросов.

Напишите программу, которая обрабатывает события вида <<добавить опрос>> и <<удалить опрос>>, и для каждого добавленного опроса сообщает его номер.

\InputFile

В первой строке входных данных находится целое число $n$ ($1 \le n \le 3\cdot 10^5$)~--- количество событий.

В каждой из следующих $n$ строк находится целое число $q_i$~--- $i$-е событие. Если $i$-е событие имеет вид <<добавить опрос>>, то $q_i$ равно $0$. Если же $i$-е событие имеет вид <<удалить опрос>>, то $q_i$ равно номеру опроса, который необходимо удалить. Гарантируется, что опрос с таким номером существует.

\OutputFile

Для каждого события добавления выведите целое число в отдельной строке~--- номер, который будет присвоен добавленному опросу.

\Examples

\begin{example}
\exmp{
9
0
0
0
2
0
0
3
1
0
}{
1
2
3
2
4
1
}%
\end{example}

\Scoring

В этой задаче \textbf{потестовая оценка}. Тесты условно разбиты на подзадачи, за полное прохождение всех тестов подзадачи начисляются соответствующие ей баллы. Подзадачи приведены в следующей таблице:

\medskip

\begin{tabular}{| c | c | c |} \hline
	№ & Ограничения & Баллы за подзадачу \\ \hline
	1 & $n \le 3000$ & 20 \\ \hline
	2 & Запросы удаления отсутствуют & 10 \\ \hline
	3 & Нет дополнитеьных ограничений & 70 \\ \hline
\end{tabular}

\end{problem}

