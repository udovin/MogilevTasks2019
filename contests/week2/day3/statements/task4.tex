\begin{problem}{Допросики}{стандартный ввод}{стандартный вывод}{4 секунды}{128 мегабайт}

Следственный комитет Контестана раследует громкое дело о похищении картины <<Пропихивание задачи в time-limit>> из Государственного музея. Сейчас следователи допрашивают свидетелей, со слов которых они узнали следующее.

Коридор, в котором находилась картина, имел длину $n$ метров. Во время ограбления в этом же коридоре проходила выставка известного своими необычными инсталляциями авангардиста Нормалевича, поэтому через каждый метр коридора находилась пружина с \textit{упругостью} $a_i$. Пронумеруем эти пружины целыми числами от $1$ до $n$ от начала до конца коридора. Картина <<Пропихивание задачи в time-limit>> находилась около пружины с номером $p$.

Следователи узнали, что грабитель пользовался пружинами, чтобы быстро покинуть коридор. А именно, если он стоял около пружины с номером $p$, то он перемещался за одну секунду к пружине $p + k + a_p$, где $k$~--- \textit{прыгучесть} грабителя. Если $p + k + a_p$ оказалось больше $n$, то это значит, что грабитель покинул помещение. Изначально грабитель стоял около картины, то есть на позиции $p$.

Следователям предстоит выяснить, за какое время грабитель покинул помещение. Однако они не знают точного значения $p$ и $k$ и хотят проверить $q$ вариантов. Для этого им понадобится программа, которая моделирует ограбление музея. Поэтому они обратились за помощью к известной компании <<Queries~\&~Requests,~Ltd.>>

Вам, как сотруднику компании, поручено написать программу для следователей, которая по заданным упругостям пружин $a_i$ определит для всех пар $(p, k)$, за какое количество секунд скрылся грабитель.

\InputFile

В первой строке входных данных находится единственное целое число $n$ ($1 \le n \le 2\cdot10^5$)~--- длина коридора в метрах.

Во второй строке задано $n$ целых чисел $a_i$ ($1 \le a_i \le n$)~--- упругость $i$-й пружины.

В третьей строке задано целое число $q$ ($1 \le q \le 2\cdot10^5$)~--- количество запросов следователя.

В следующих $q$ строках задано по два целых числа $p$ и $k$ ($1 \le p, k \le n$)~--- описание очередного запроса.

\OutputFile

Выведите $q$ чисел, по одному числу в каждой строке. $i$-е число обозначает ответ на $i$-й запрос.

\Examples

\begin{example}
\exmp{
3
1 1 1
3
1 1
2 1
3 1
}{
2
1
1
}%
\end{example}

\Notes

Разберем пример из условия:

В первом запросе за первую секунду грабитель прыгает с позиции $1$ на позицию $1 + 1 + 1 = 3$, за вторую~--- на позицию $3 + 1 + 1 = 5$, то есть покидает помещение.

Во втором и третьем запросах $p$ становится больше $n$ уже после первого прыжка.

\Scoring
В этой задаче \textbf{оценка по подзадачам}. Тесты условно разбиты на подзадачи, за полное прохождение всех тестов подзадачи начисляются соответствующие ей баллы. Если какой-то тест подзадачи не пройден, баллы за нее не начисляются. Подзадачи приведены в следующей таблице:

\medskip

\begin{tabular}{| c | c | c |} \hline
	№ & Ограничения & Баллы за подзадачу \\ \hline
	1 & $n, q \le 10^3$ & $10$ \\ \hline
	2 & $a_i = 1$ & $10$ \\ \hline
	3 & $n \le 6000$ & $40$ \\ \hline
	4 & Нет дополнительных ограничений & $40$ \\ \hline
\end{tabular}

\end{problem}

