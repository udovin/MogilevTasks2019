\documentclass[12pt]{article}
\usepackage[utf8]{inputenc}
\usepackage[english,russian]{babel}
\usepackage{verbatim}
\usepackage{listings}
\usepackage{geometry}
\usepackage{graphicx}
\usepackage{indentfirst}

\geometry{a4paper, top=40pt, bottom=70pt, left=40pt, right=40pt}

\newcommand{\TimeLimit}[1]{Ограничение по времени: \texttt{#1 сек} \newline}
\newcommand{\MemoryLimit}[1]{Ограничение по памяти: \texttt{#1 MiB} \newline}
\newcommand{\InputFile}[1]{Входной файл: \texttt{#1} \newline}
\newcommand{\OutputFile}[1]{Выходной файл: \texttt{#1} \newline}
\newcommand{\Problem}[5]{
	\section{#1}
	\noindent
	\TimeLimit{#2}
	\MemoryLimit{#3}
	\InputFile{#4}
	\OutputFile{#5}
}
\newcommand{\EndProblem}{\clearpage}
\newcommand{\Legend}{}
\newcommand{\Input}{\subsection*{Входные данные}}
\newcommand{\Output}{\subsection*{Выходные данные}}
\newcommand{\Samples}{\subsection*{Примеры тестов}}
\newcommand{\Scoring}{\subsection*{Частичная оценка}}
\newcommand{\BeginTests}{
	\noindent
	\begin{tabular}[pos]{|l|l|}
	\hline
	Входные данные & Выходные данные \\
	\hline
}
\newcommand{\Test}[3]{
	\lstinputlisting{#1/#2} & \lstinputlisting{#1/#3} \\
	\hline
}
\newcommand{\EndTests}{\end{tabular}}

\renewcommand{\thesection}{\Alph{section}}

\begin{document}
\Editorial{Корневые LCA запросы}{3}

Подвесим дерево за вершину $1$, затем найдем количество вершин $cnt[v]$ в каждом поддереве $v$, а также предпосчитаем $lca$.
Теперь рассмотрим несколько случаев:
\begin{itemize}
	\item вершина $a$ находится в поддереве $c$, вершина $b$ находится в поддереве $c$,
	\item вершина $a$ не находится в поддереве $c$, вершина $b$ находится в поддереве $c$,
	\item вершина $a$ находится в поддереве $c$, вершина $b$ не находится в поддереве $c$,
	\item вершина $a$ не находится в поддереве $c$, вершина $b$ не находится в поддереве $c$.
\end{itemize}

В случае $1$ все просто: если $lca(a, b) = c$, то мы можем подвесить дерево за все
вершины, кроме вершин, которые лежат в поддереве вершины $c$, где лежат вершины $a$ и $b$.
Т.е. ответом будет $n - cnt[a] - cnt[b]$.
Иначе ответ равен $0$, т.к. вершина $c$ не является наименьшим общим предком и лежит "выше" $lca$.

В случаях $2$ или $3$ мы не можем подвесить за вершины, которые находятся "выше" вершины $c$ и которые лежат соответственно в поддереве $b$ или $a$.
Ответом будет $cnt[c] -$ количество вершин в поддереве, где лежат вершины $b$ или $a$ соответственно. Для того, чтобы найти это количество, нужно
взять $lca(c, b или a)$, однако при подсчете $lca$ мы возьмем не сам наименьший общий предок, а его потомок в поддереве, где лежат $b$ или $a$ соответственно.

В случае $4$ ответ $0$, т.к.вершины $a$ и $b$ лежат выше вершины $c$.

\EndEditorial
\Editorial{Ленивый гонец}{1}

Переберем город, в котором мы начнем наш путь, за $O(n^2)$ просчитаем ответ согласно правилам и возьмем минимум.
\EndEditorial
\Editorial{Выборы}{2}

Заметим, что нам не выгодно голосовать за кандидата $i$, если дорога, которая ведет из города $i$ в его предок, не является проблемной.
Тогда запустим обход в глубину и на выходе из вершины вернем $true$, если мы выбрали кого-то в поддереве или какая-либо дорога к детям вершины является проблемной.

\EndEditorial

\end{document}
