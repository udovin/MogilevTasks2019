\documentclass[12pt]{article}
\usepackage[utf8]{inputenc}
\usepackage[english,russian]{babel}
\usepackage{verbatim}
\usepackage{listings}
\usepackage{geometry}
\usepackage{graphicx}
\usepackage{indentfirst}

\geometry{a4paper, top=40pt, bottom=70pt, left=40pt, right=40pt}

\newcommand{\TimeLimit}[1]{Ограничение по времени: \texttt{#1 сек} \newline}
\newcommand{\MemoryLimit}[1]{Ограничение по памяти: \texttt{#1 MiB} \newline}
\newcommand{\InputFile}[1]{Входной файл: \texttt{#1} \newline}
\newcommand{\OutputFile}[1]{Выходной файл: \texttt{#1} \newline}
\newcommand{\Problem}[5]{
	\section{#1}
	\noindent
	\TimeLimit{#2}
	\MemoryLimit{#3}
	\InputFile{#4}
	\OutputFile{#5}
}
\newcommand{\EndProblem}{\clearpage}
\newcommand{\Legend}{}
\newcommand{\Input}{\subsection*{Входные данные}}
\newcommand{\Output}{\subsection*{Выходные данные}}
\newcommand{\Samples}{\subsection*{Примеры тестов}}
\newcommand{\Scoring}{\subsection*{Частичная оценка}}
\newcommand{\BeginTests}{
	\noindent
	\begin{tabular}[pos]{|l|l|}
	\hline
	Входные данные & Выходные данные \\
	\hline
}
\newcommand{\Test}[3]{
	\lstinputlisting{#1/#2} & \lstinputlisting{#1/#3} \\
	\hline
}
\newcommand{\EndTests}{\end{tabular}}

\renewcommand{\thesection}{\Alph{section}}

\begin{document}
\Editorial{Дроби}{2}

Найдем пару знаменателей $x$ и $y$ такую, что $x < y$ и $x \cdot y = n$ и $GCD(x, y) = 1$.
Числителями будут $a$ и $b$ соответсвенно. Получим, что:
$\frac{a \cdot y + b \cdot x}{x \cdot y} = \frac{n - 1}{n}$, или $a \cdot y + b \cdot x = n - 1$.
Чтобы найти такие $a$ и $b$, просто зафиксируем число $a$. Выражение $(a \cdot y) mod x$ принимает
может принимать значения от $1$ до $x - 1$, потому что $a \cdot y$ не делится на $x$ так как числа 
$x$ и $y$ взаимнопростые. Тем не менее, существует такое $a$, что $a \cdot y + 1$ делится на $x$. Возьмем это
число и вычислим $b = \frac{n - 1 - c \cdot y}{x}$. Так как $n$ и $a \cdot y + 1$ делятся на $x$, то получим,
что число $b$ является целым положительным числом. Мы нашли решение задачи. Заметим, что подойдет любая
пара $x$ и $y$ удовлетворяющая условиям.

А что, если такой пары не нашлось? Заметим, что это число $p^{m}$, где $p$~--- некоторое простое число.
Если привести дроби к общему знаменателю получим, что числитель должен делится на $p$, однако $n - 1$ не
делится на $p$, а значит решения не существует.

\EndEditorial
\Editorial{Дома в Colorville}{4.33}

Теорема Пойа:

$K = \frac{1}{|G|} \sum_{p \in G} a^{C(p)}$, где $K$~---
количество классов эквивалентности, $G$~--- множество инвариантных перестановок,
$a$~--- количество принимаемых значений одним элементом,
$C(p)$~--- количество циклов в перестановке $p$.

Для нашей задачи: $a = c^{n^2}$, инвариантные перестановки~--- все циклические сдвиги,
$K$~--- ответ на задачу.

Циклы в перестановке можно искать обходом в глубину, а можно заметить одну интересную
формулу: $C(p_m) = gcd(n, m)$, где $m$~--- номер циклического сдвига, $n$~--- размер перестановки.

Получим следующую формулу: $\frac{1}{m} \sum_{i = 1}^{m} (c^{n^2})^{gcd(i, m)}$.

\EndEditorial
\Editorial{Абсолют}{2}

Число $k$ встречается $[\frac{n}{k}]$ раз. Воспользуемся этим: подсчитав значение на текущем отрезке
$[k, [\frac{n}{[\frac{n}{k}]}]]$, найдем такое минимальное число, что встречается $[\frac{n}{k}] + 1$ раз.
Пройдем по всем таким отрезкам, их порядка $O(\sqrt n)$.

\EndEditorial
\Editorial{Цепочность числа}{2}

Нужно прочитать условие иначе. Чтобы выполнялось условие, нужно, чтобы $n$ давало остаток $1$ от деления на $k + 1$.
Другими словами $n - 1$ должно делится на $k + 1$. Теперь мы хотим найти все такие числа, которые делится на первые $k$ делителей:
$\frac{n - 1}{lcm(1, ..., k)} - \frac{n - 1}{lcm(1, ..., k + 1)}$. Последнее наблюдение заключается в том, что $lcm(1, 2, ...)$
растет очень быстро.

\EndEditorial
\end{document}
