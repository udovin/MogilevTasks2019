\documentclass[12pt]{article}
\usepackage[utf8]{inputenc}
\usepackage[english,russian]{babel}
\usepackage{verbatim}
\usepackage{listings}
\usepackage{geometry}
\usepackage{graphicx}
\usepackage{indentfirst}

\geometry{a4paper, top=40pt, bottom=70pt, left=40pt, right=40pt}

\newcommand{\TimeLimit}[1]{Ограничение по времени: \texttt{#1 сек} \newline}
\newcommand{\MemoryLimit}[1]{Ограничение по памяти: \texttt{#1 MiB} \newline}
\newcommand{\InputFile}[1]{Входной файл: \texttt{#1} \newline}
\newcommand{\OutputFile}[1]{Выходной файл: \texttt{#1} \newline}
\newcommand{\Problem}[5]{
	\section{#1}
	\noindent
	\TimeLimit{#2}
	\MemoryLimit{#3}
	\InputFile{#4}
	\OutputFile{#5}
}
\newcommand{\EndProblem}{\clearpage}
\newcommand{\Legend}{}
\newcommand{\Input}{\subsection*{Входные данные}}
\newcommand{\Output}{\subsection*{Выходные данные}}
\newcommand{\Samples}{\subsection*{Примеры тестов}}
\newcommand{\Scoring}{\subsection*{Частичная оценка}}
\newcommand{\BeginTests}{
	\noindent
	\begin{tabular}[pos]{|l|l|}
	\hline
	Входные данные & Выходные данные \\
	\hline
}
\newcommand{\Test}[3]{
	\lstinputlisting{#1/#2} & \lstinputlisting{#1/#3} \\
	\hline
}
\newcommand{\EndTests}{\end{tabular}}

\renewcommand{\thesection}{\Alph{section}.}

\begin{document}
\Editorial{Новый ноутбук}{1.33}

На $60$ баллов перебираем левую и правую границу, считаем модуль, если надо, то увеличиваем ответ на $1$.

На $100$ баллов мы будем для каждого $i$, $(1 \le i \le n)$ считать количество таких чисел, что их модуль в точности равен $j$, ($0 \le j \le m - 1$).
Тогда для заметим переход из $dp[i][j]$ в $dp[i + 1][(j * 10 + s[i]) mod m]$, где $s[i]$~--- $i$-ый символ строки $s$.
Также стоит обратить внимание на то, что мы можем хранить только последний слой динамики, в противном случае мы превышаем ограничение по памяти.

\EndEditorial

\Editorial{Новый ноутбук}{2.33}

На $30$ баллов перебираем все возможные варианты строк.

Затем заметим, что второй символ в строке $a_i$ не влияет на конечный результат.

На $100$ баллов мы будем для каждого $i$, $(1 \le i \le n)$ считать количество таких строк, что их $i$-ый символ в точности равен $j$-ому в алфовитном порядке, ($0 \le j \le 26$).
Тогда для заметим переход из $dp[i][j]$ мы можем перейти в клетки согласно правилам, описаным во входных данных. Т. е. из $dp[i][j]$ мы можем перейти в $dp[i + 1][k]$,
где $j$~--- первый символ строки $a_t$, а $k$~--- символ $b_t$.
Также стоит обратить внимание на то, что возможно несколько одинаковых пар $a_i, b_i$. В таком случае лучше сжать значения.

\EndEditorial

\Editorial{Супер М}{2.33}

На $60$ баллов подвешиваем дерево за каждую вершину и считаем ответ.

Заметим, что что-то похожее было на области.

На $100$ баллов мы запустим 2 поиска в глубину. Подвесим дерево за вершину 1.
Сначала мы посчитаем количество атакуемых городов и суммарное расстояние до них в поддереве для каждой вершины.
Затем запустим 2 поиск в глубину и будем сохранять ответ и передавать в поиск глубину количество атакуемых городов и суммарное расстояние до них для всех вершин, кроме вершин поддерева, в которое мы переходим.

\EndEditorial

\end{document}
