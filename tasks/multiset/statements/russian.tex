\Problem{Мультимножество}{1}{64}{multiset.in}{multiset.out}

\Legend
У вас есть q запросов и мультимножество $A$, изначально содержащее только число $0$. Запросы бывают трёх видов:
\begin{itemize}
	\item ``+ x''~--- добавить в мультимножество $A$ число $x$.
	\item ``- x''~--- удалить одно вхождение числа $x$ из мультимножества $A$. Гарантируется, что хотя бы одно число $x$ в этот момент присутствует в мультимножестве.
	\item ``? x''~--- вам даётся число $x$, требуется вычислить $\max_{y \in A}{x \oplus y}$, то есть максимальное значение побитового исключающего ИЛИ (также известно как XOR) числа $x$ и какого-нибудь числа $y$ из мультимножества $A$.
\end{itemize}

Мультимножество~--- это множество, в котором разрешается несколько одинаковых элементов.

\Input
В первой строке входных данных содержится число $q$ ($1 \le q \le 200\,000$)~--- количество запросов, которые требуется обработать Василию.

Каждая из последующих $q$ строк входных данных содержит один трёх символов ``+'', ``-'' или ``?'' и число $x_i$ ($1 \le x_i \le 10^9$).
Гарантируется, что во входных данных встречается хотя бы один запрос ``?''.

Обратите внимание, что число $0$ всегда будет присутствовать в мультимножестве.

\Output
На каждый запрос типа ``?'' выведите единственное целое число~--- максимальное значение побитового исключающего ИЛИ для числа $x_i$ и какого-либо числа из мультимножества $A$.

\Samples
\BeginTests
\Test{tasks/multiset/tests/samples}{01}{01.a}
\EndTests

\Scoring
\begin{itemize}
	\item $q \le 1\,000$~--- 20 баллов.
	\item $q \le 5\,000$~--- 60 баллов
	\item без ограничений~--- 100 баллов.
\end{itemize}

\EndProblem
