\Problem{Цепочность числа}{2}{256}{numstreak.in}{numstreak.out}

\Legend
Ни для кого не секрет, что школьники Байтландии обожают математику. Апполинария - отличница
школы №2019, а математика~--- её любимый предмет!

На последнем занятий Тринидад Итобагович ввел понятие цепочности числа. Цепочность числа $n$~---
это такое минимальное целое положительное значение $s$, что $s + 1$ делит число $n + s$ с остатком.
Например, для числа $13$:
\begin{itemize}
	\item $s = 1$: $14$ делится на $2$,
	\item $s = 2$: $15$ делится на $3$,
	\item $s = 3$: $16$ делится на $4$,
	\item $s = 4$: $17$ \textbf{не} делится на $5$.
\end{itemize}

Тогда цепочность числа $13$ равна $4$.

Таким же образом получим, что цепочность числа $120$ равна $1$, ведь $121$ не делится на $2$.

Тринидад Итобагович пристально следит за успехами своих учеников. Апполинария, как отличница,
сразу получила на дом задание повышенной трудности: найти количество таких чисел, не больших $n$,
что их цепочность равна в точности $k$. Помогите Апполинарии решить домашнее задание.

\Input
Первая строка входного файла содержит два целых числа $n$ и $k$ ($1 \le n \le 10^{18}$, $1 \le k \le 10^{18}$)
~--- интервал и то, чему должна равняться цепочность соответственно. Числа в строке разделены пробелом.

\Output
Единственная строка выходного файла должна содержать ответ на задачу~--- количество чисел, не больших $n$
таких, что их цепочность равна $k$.

\Samples
\BeginTests
	\Test{tasks/numstreak/tests/samples}{1.in}{1.out}
	\Test{tasks/numstreak/tests/samples}{2.in}{2.out}
\EndTests

В первом примере единственное такое число~--- $7$.

\Scoring
\begin{itemize}
	\item $n \le 10^7$, $k \le 42$~--- 60 баллов,
	\item $n \le 10^{18}$, $k \le 42$~--- 90 баллов,
	\item $n \le 10^{18}$, $k \le 10^{18}$~--- 100 баллов.
\end{itemize}

\EndProblem
