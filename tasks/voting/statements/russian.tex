\Problem{Выборы}{1}{64}{voting.in}{voting.out}

В городе, где живет Анатолий, скоро состоятся выборы в Городскую Думу.

Всего в городе есть $n$ районов и $n - 1$ двусторонняя дорога. При этом известно, что из любого района существует путь по дорогам в любой другой район. Пронумеруем все районы в городе некоторым образом целыми числами от $1$ до $n$ включительно. Для каждой дороги жители определили, является ли она проблемной или нет. Проблемная дорога~--- это дорога, которая нуждается в ремонте.

На выборы баллотируются $n$ кандидатов. Пронумеруем всех кандидатов некоторым образом целыми числами от $1$ до $n$ включительно. Если кандидат с номером $i$ будет избран в Городскую Думу, то он выполнит ровно одно обещание — отремонтировать все проблемные дороги на пути от района с номером $i$ до района с номером $1$, в котором расположена Городская Дума.

Помогите Анатолий и найдите множество кандидатов, после избрания которых в Городскую Думу все проблемные дороги в городе будут отремонтированы. Если же существует несколько таких множеств, следует выбрать множество, состоящее из минимального количества кандидатов.

\Input
В первой строке задано одно целое число $n$ ($1 \le n \le 10^{5}$)~--- количество районов в городе.

Далее следует $n - 1$ строка. Каждая строка содержит описание дороги в виде трех целых положительных чисел $x_i$, $y_i$, $t_i$ ($1 \le x_i, y_i \le n$, $1 \le t_i \le 2$)~--- районы, которые соединяет $i$-я двусторонняя дорога, и тип дороги. Если $t_i$ равно единице, то $i$-я дорога не является проблемной; если $t_i$ равно двум, то $i$-я дорога является проблемной.

Гарантируется, что если представить город в виде графа дорог, граф будет являться деревом.

\Output
В первой строке выведите одно целое неотрицательное число $k$~--- минимальный возможный размер искомого множества.

Во второй строке выведите $k$ целых чисел $a_1, a_2, ..., a_k$~--- номера кандидатов, которые образуют искомое множество. Если существует несколько решений, то разрешается вывести любое.

\Samples
\BeginTests
	\Test{tasks/voting/tests/samples}{01}{01.a}
    \Test{tasks/voting/tests/samples}{02}{02.a}
    \Test{tasks/voting/tests/samples}{03}{03.a}
\EndTests

\Scoring
\begin{itemize}
	\item $n \le 20$~--- 30 баллов,
	\item $n \le 5000$~--- 60 баллов.
\end{itemize}

\EndProblem
