\Problem{Инверсии}{2}{256}{inversions.in}{inversions.out}

\Legend
Сложность задачи: $5$.

Рассмотрим последовательность из $n$ целых чисел от $1$ до $k$ (включительно). Некоторые из целых чисел отсутствуют и заменяются на $0$.
Инверсия~--- это пара значений $a_i$ и $a_j$ в последовательности, где $i < j$, но $a_i > a_j$. Какое максимальное количество
инверсий можно получить, если все пропущенные целые числа находятся между $1$ и $k$ включительно?

\Input
В первой строке задано два целых числа $n$ ($1 \le n \le 2 \cdot 10^5$) и $k$ ($1 \le k \le 100$), где $n$~---
длина последовательности, а $k$~--- максимальное значение элементов последовательности.

Каждая из следующих $n$ строк cодержит одно целое число $x$ ($0 \le x \le k$)~--- последовательность по порядку, с $0$
представляющими пропущенные значения.

\Output
Выведите единственное целое число, которое является максимально возможным числом инверсий.

\Samples
\BeginTests
	\Test{tasks/inversions/tests/samples}{01}{01.a}
	\Test{tasks/inversions/tests/samples}{02}{02.a}
	\Test{tasks/inversions/tests/samples}{03}{03.a}
\EndTests

\Scoring
\begin{itemize}
	\item $n, k \le 10$~--- 20 баллов.
	\item $n \le 100$~--- 60 баллов.
	\item без дополнительных ограничений~--- 100 баллов.
\end{itemize}

\EndProblem
