Вам дано натуральное число $n$.

Найдите последовательность дробей вида $\frac{a_i}{b_i}$, $i = 1...k$
(где $a_i$ и $b_i$ натуральные числа) для некоторого числа $k$ такую, что:

1. $b_i$ делит $n$, $1 \le b_i \le n$ для $i = 1...k$,
2. $1 \le a_i \le b_i$ для $i = 1...k$,
3. $\sum_{i = 1}^{k} \frac{a_i}{b_i} = 1 - \frac{1}{n}$.

Входные данные

В единственной строке содержится одно целое число $n$ ($2 \le n \le 10^9$).

Выходные данные

В первой строке выходных данных выведите ``YES'', если существует такая последовательность дробей, в противном случае выведите ``NO''.

Если такая последовательность существует, то следующие строки должны содержать описание данной последовательности в следующем формате.

Вторая строка выходных данных должна содержать число $k$ ($1 \le k \le 100\,000$) --- количество элементов в последовательности.
Гарантируется, что если такая последовательность существует, то существует последовательность, длина которой не превышает $100\,000$.

В следующих $k$ строках выходных данных должны содержаться дроби последовательности в виде двух чисел $a_i$ и $b_i$ разделенных пробелом в каждой строке.

Примечания

Во втором примере последовательность $\frac{1}{2}$, $\frac{1}{3}$ такая, что $\frac{1}{2} + \frac{1}{3} = 1 - \frac{1}{6}$.
