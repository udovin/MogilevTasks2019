\Problem{Дисконтные карты}{1}{64}{nthcard.in}{nthcard.out}

\Legend
Мисье Антонио страстно любит путешествовать между аэропортами некоторых стран. Конкретнее,
наш новый знакомый посещает аэропорты трех стран: Байторибо, Байтобаджо и Байтвиль.

Мисье Антонио~--- счастливый обладатель $n$ дисконтных карт различных авиакомпаний. Каждая из
них дает ему шанс пролететь со скидкой 99.9\% процентов на рейсе. У него запланировано $k$
рейсов, на каждый из которых ему нужна какая-то из его карточек.

Иногда он вынужден долго искать в кошельке нужную карточку. Это отнимает его драгоценное
время и сбивает жизнь с привычного ритма.

Карточки в кошельке мисье Антонио хранятся одной стопкой. Он хочет, чтобы каждый раз, когда
у него возникает необходимость воспользоваться картой, наверху стопки оказывалась именно та,
которая ему нужна. При этом для него не проблема после использования карты убрать её в
определённое место стопки.

Зная, в каком порядке будут использоваться карты, несложно складывать их так, чтобы сверху
всегда оказывалась нужная. Мисье Антонио мог бы решить эту задачу, но ему не хочется. Так что
займитесь-ка этим сами.

\Input
В первой строке входного файла записано два числа~--- $n$ и $k$ ($1 \le n, k \le 10^5$)~--- общее количество
карточек у Антонио в кошельке и количество раз, которое Антонио использует свои карточки.
Карточки пронумерованы целыми числами от $1$ до $n$.

Вторая строка содержит $n$ целых чисел $a_i$ ($1 \le a_i \le 10^9$), разделенных одиночными пробелами~---
номера карточек, перечисленные в порядке их использования.

\Output
Выведите $k+1$ строку.

В первой строке должны находиться записанные через пробел $n$ чисел от $1$ до $n$~--- порядок,
в котором карты требуется расположить изначально, от самой верхней до самой нижней.

В $(i+1)$-й строке должно содержаться одно число~--- сколько карточек будет располагаться
над ai-й сразу после её возвращения в кошелёк.

Если существует несколько возможных решений, выведите любое.

\Samples
\BeginTests
\Test{tasks/nthcard/tests/samples}{01}{01.a}
\Test{tasks/nthcard/tests/samples}{01}{01.a}
\EndTests

\Scoring
\begin{itemize}
	\item каждая карточка используется по разу,
	$1 \le n = k \le 10$~--- 20 баллов.
	\item $k \le 1000$~--- 40 баллов
	\item $k \le 5000$~--- 60 баллов
	\item без ограничений~--- 100 баллов.
\end{itemize}

\EndProblem
