\Problem{Задача коммивояжера}{1}{64}{inputX.txt}{outputX.txt}

\Legend
Вам дан полный граф состоящий из $n$ вершин. В каждой вершине записано некоторое целое число $a_i$. Ориентированное ребро,
присоединяющее вершину $i$ к $j$ имеет вес $w_{ij}$ ($i$-строка и $j$-й столбец матрицы). Стоимость перехода от $i$-й
вершины к $j$-й определяется по формуле: $(w_{ij} + k \& a_{i}) \oplus a_{j}$, где $k$~--- количество уже посещенных
вершин (не включая $j$-ю). Требуется найти такой путь, что каждая вершина будет посещена ровно один раз, а стоимость пути
будет минимальна.

\Input
В первой строке входного файла записано одно целое число $n$~--- размер графа.

В следующей строке записано $n$ чисел $a_i$.

В следующих $n$ строках записана матрица смежности для данного графа.

\Output
В единственной строке выведите $n$ чисел $p_i$ ($1 \le p_i \le n$)~--- перестановку вершин.

\Samples
\BeginTests
\Test{tasks/salesman/tests/samples}{01}{01.a}
\Test{tasks/salesman/tests/samples}{02}{02.a}
\EndTests

В первом примере суммарная стоимость пути: $(w_{12} + 1 \& a_{1}) \oplus a_{2} = (4 + 1 \& 1) \oplus 2 = 7$.

\Scoring
Если ваше решение находит путь стоимостью $A$, а лучшее решение другого участника $B$, то вы получите
$10 \cdot (B/A)^3$ баллов за тест.

\EndProblem
