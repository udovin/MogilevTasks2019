\Problem{``Боброжуи-0xFF''}{1}{64}{inputX.txt}{outputX.txt}

\Legend
\textit{Один ``Боброжуй-0xFF'' хорошо, а два лучше.}

``Боброжуй-0xFF''~--- не совсем совершенная машина. Точнее была совершенной до тех пор, пока не появилась
еще одна. Вся проблема в том, что Боброжуи-0xFF \textbf{конфликтуют} между собой, если действуют по
\textbf{одному алгоритму}.

В городе М. было решено поместить $n$ Боброжуев-0xFF. Для этого выбрали $n$ участков, за каждый из которых
должен отвечать один Боброжуй-0xFF. Некоторые участки находятся близко друг другу (являются соседними),
поэтому в этих местах \textbf{нельзя} помещать Боброжуев-0xFF действующих по одинаковому алгоритму, иначе они
могут сломаться.

Разработка уникального алгоритма требует много времени и денег, поэтому необходимо минимизировать затраты.
Определите минимальное количество алгоритмов, которое понадобится разработать, чтобы ни один Боброжуй-0xFF
не конфликтовал.

\Input
В первой строке входного файла записано два целых числа $n$ и $m$~--- количество Боброжуев-0xFF и количество
соседних участков.

В следующих $m$ строках записано по два целых числа $x$ и $y$~--- номера соседних участков.

\Output
В первой строке выведите одно число $k$~--- количество алгоритмов, которые необходимо разработать.

В следующей строке выведите $n$ целых чисел $a_i$ ($1 \le a_i \le n$)~--- номера алгоритмов, по которым должен
работать Боброжуй-0xFF, находящийся в $i$-м участке.

\Samples
\BeginTests
\Test{tasks/graphcoloring/tests/samples}{01}{01.a}
\EndTests

\Scoring
Если вы нашли решение с использованием $k$ алгоритмов, а лучшее решение другого участника с использованием
$k_{best}$, то вы получите $10 \cdot (k_{best}/k)^3$ баллов за тест.

Если в вашем решении некоторые Боброжуи-0xFF могут конфликтовать, вы получите $0$ баллов.

\EndProblem
