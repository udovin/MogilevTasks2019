\Problem{Сжатие}{1}{64}{archivation.in}{archivation.out}

\Legend

Антон терпеть не может длинные строки, поэтому предпочитает сжимать их следующим образом. Имеется $q$ операций сжатия строк. Операции можно применять в любом порядке и каждую можно применить произвольное количество раз. Операция с номером $i$ задаётся строкой $a_i$ длины два и строкой $b_i$ длины один. Все строки $a_i$ различны.

У Антона есть строка $s$, к которой можно применить $i$-ю операцию, только если первые два символа этой строки совпадают со строкой $a_i$. Применение операции заключается в отбрасывании этих двух символов и приписывании в начало строки $b_i$.

Несложно заметить, что применение любой операции уменьшает длину строки $s$ ровно на $1$. Также для некоторых наборов операций возможно существование строки, сжатие которой невозможно, потому что первые две буквы не совпадают ни с одной из строк $a_i$.

Антон хочет начать со строки длины $n$ и применить $n - 1$ операцию, чтобы в итоге получить строку «$a$». Сколько существует подходящих строк, из которых можно получить «$a$»? Не забывайте, что ответ может берется по модулю $10^9 + 7$, т. к. иногда он может быть довольно большим.

\Input
В первой строке входного файла записаны два числа $n$ и $q$ ($2 \le n \le 5000, 1 \le q \le 17576$)~--- начальная длина строки и количество доступных операций соответственно.

В следующих $q$ строках даны описания операций. В $i$-й из них записаны строки $a_i$ и $b_i$ ($|a_i| = 2, |b_i| = 1$).

\Output
Выведите количество строк длины $n$, таких что Антон сможет перевести в строку «$a$», используя только имеющиеся $q$ операций.

\Samples
\BeginTests
	\Test{tasks/archivation/tests/samples}{01}{01.a}
	\Test{tasks/archivation/tests/samples}{02}{02.a}
	\Test{tasks/archivation/tests/samples}{03}{03.a}
\EndTests

\Scoring
\begin{itemize}
	\item $n \le 4$, $q \le 10$~--- 30 баллов,
	\item $n \le 200$, $q \le 200$~--- 60 баллов,
	\item без дополнительных ограничений~--- 100 баллов.
\end{itemize}

\EndProblem
