\Problem{Ваучеры Фибоначчи}{1}{64}{fibonacci.in}{fibonacci.out}

Школа организует ярмарку в память о Фибоначчи. Джонни отвечает за магазин подарков, в котором вы можете
платить только с помощью специальных ваучеров, чьи номиналы являются числами Фибоначчи. Джонни с трудом обрабатывает
такие странные значения и решил принимать только точные платежи~--- ровно $k$ ваучеров, не обязательно
разных номиналов. Теперь ему нужно установить цены~--- в магазине подарков есть $n$ разных предметов, а Джонни хочет
установить разные цены для каждого из них. Иногда существует много способов заплатить одну цену, в таком случае Джонни
рассчитывает эту цену только один раз. Он рассчитал все цены и теперь хочет проверить свои расчеты. Делать это довольно
быстро, достаточно назвать последнюю, $n$-ую цену. Помогите Джонни~--- напишите программу, которая, по заданным $n$ и $k$,
вычисляет $n$-я наименьшая цена, которую можно заплатить, используя ровно $k$ ваучеров.

\Input
Единственная строка ввода содержит два целых числа $k$ и $n$ ($1 \le k \le 100$, $1 \le n \le 10^{18}$), разделенных одним пробелом.

\Output
В единственной строке вы должны вывести одно целое число~--- $n$-я наименьшая цена, которая может
оплачиваться $k$ (не обязательно разными) ваучерами, чьи номиналы являются числами Фибоначчи, при условии, что это
число не больше $10^{18}$, или ``\texttt{No}'', если оно больше $10^{18}$.

\Samples
\BeginTests
	\Test{tasks/fibonacci/tests/samples}{01}{01.a}
\EndTests

\Scoring
\begin{itemize}
	\item $k = 1$~--- 20 баллов.
	\item $k = 2$, $n \le 100$~--- 20 баллов.
	\item без дополнительных ограничений~--- 60 баллов.
\end{itemize}

\EndProblem
