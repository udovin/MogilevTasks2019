\Problem{Новый ноутбук}{1}{64}{notebook.in}{notebook.out}

\Legend

Дима хочет купить новый ноутбук. Он уже посчитал сколько ему нужно денег, чтобы купить новый ноутбук. Оставив калькулятор на полу, он пошёл просить у родителей денег. В это время его маленький брат Никита пришёл в комнату и начал нажимать на случайные кнопки калькулятора. К сожалению Дима уже забыл сумму, посчитанную на калькуляторе. Единственное он помнит, что сумма была кратна $m$.

Вам задана строка $s$ состоящая из цифр (число, которое отображалось на экране калькулятора после того как Никита понажимал на случайные кнопки). Ваша задача определить количество подстрок кратных $m$. Подстрока может начинаться с нуля.

Подстрокой строки называется непустая последовательность подряд идущих символов.

\Input
В единственной строке входного файла задана строка $s$ и модуль $m$ ($1 \le |s|, m \le 10^5$, $|s| * m \le 10^6$).

\Output
Выведите целое число $ans$ — количество подстрок строки $s$ кратных $m$.

\Samples
\BeginTests
	\Test{tasks/notebook/tests/samples}{01}{01.a}
	\Test{tasks/notebook/tests/samples}{02}{02.a}
	\Test{tasks/notebook/tests/samples}{03}{03.a}
\EndTests

\Scoring
\begin{itemize}
	\item $|s| \le 100$, $m \le 100$~--- 30 баллов,
	\item $n, m \le 5000$~--- 60 баллов,
	\item без дополнительных ограничений~--- 100 баллов.
\end{itemize}

\EndProblem
