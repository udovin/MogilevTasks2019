\Problem{Супер М}{1}{64}{superm.in}{superm.out}

\Legend

На самом деле Даник~--- не совсем обычное чудовище. Он представляет собой скрытую личность Супер М, одного из главных супергероев Могилева. Могилев~--- это страна, состоящая из $n$ городов, соединенных $n - 1$ дорогами. Каждая дорога соединяет ровно два различных города, и вся дорожная система разработана так, что можно добраться от любого города до любого другого города, используя только данные дороги.

На $m$ городов напали злые люди. И вот, Даник... то есть, Супер М должен немедленно направиться в каждый из атакуемых городов, чтобы прогнать злых людей. Супер М может перемещаться между городами только используя данные дороги. Более того, перемещение по любой из дорог занимает у неё ровно один бублик (единица времени, используемая в Могилеве).

Однако, сейчас Супер М не в Могилеве — он посещает тренировочный лагерь, расположенный в близлежащей стране Барановичи. К счастью, в Барановичи есть особый прибор, позволяющий мгновенно телепортировать девушку из Баранович в любой город Могилева. Обратный путь слишком долог, так что в рамках данной задачи телепортация используется ровно один раз.

Вам следует помочь Супер М и вычислить город, в который он должна телепортироваться в начале, чтобы завершить свою работу за минимальное время (измеряемое в бубликах). Также сообщите Данику это время, чтобы он могла распланировать дорогу обратно в Барановичи.

\Input

В первой строке входных данных записаны два целых числа $n$ и $m$ ($1 \le m \le n \le 2 \cdot 10^5$)~--- количество городов в Могилеве, и количество атакуемых городов, соответственно.

Далее следует $n - 1$ строка, описывающая систему дорог. Каждая строка состоит из двух номеров городов $u_i$ и $v_i$ ($1 \le u_i, v_i \le n$)~--- концы очередной дороги.

Последняя строка содержит $m$ различных целых чисел — номера атакуемых городов. Эти числа даны в произвольном порядке.

\Output

Сперва выведите номер города, куда надо телепортироваться Супер М. Если оптимальных ответов несколько, выведите город с наименьшим номером.

В следующей строке выведите $n$ чисел, $i$-ое из которых - наименьшее возможное время, необходимое для того, чтобы распугать всех людей в атакуемых городах, если Супер М телепортируется в город $i$.

\Samples
\BeginTests
	\Test{tasks/superm/tests/samples}{01}{01.a}
	\Test{tasks/superm/tests/samples}{02}{02.a}
\EndTests

\Scoring
\begin{itemize}
	\item $n \le 20$~--- 20 баллов,
	\item $n \le 5000$~--- 60 баллов,
	\item без дополнительных ограничений~--- 100 баллов.
\end{itemize}

\EndProblem
